% !TEX TS-program = xelatex
% !TEX encoding = UTF-8
\documentclass[a4,10pt]{article}
\usepackage{exam}
%\usepackage{pslatex}
\usepackage{graphicx}
\usepackage[toc,page]{appendix}
\usepackage[framed,numbered,autolinebreaks,useliterate]{mcode}
\input{listingstyle}
\renewcommand*{\lstlistingname}{Code [\textsc{\scriptsize{MATLAB}}]}
\usepackage{mathptmx}	% Use the Postscript Times font
\usepackage[FIGBOTCAP,normal,bf,tight]{subfigure}
\usepackage[normal,bf,tight]{subfigure}
\usepackage[dvips,light,first,bottomafter]{draftcopy}
\draftcopyName{Sample, contains no OUO}{70}
\usepackage{amsmath,amssymb,amsthm}
\usepackage{kotex}
\usepackage{multirow}
\usepackage{multicol}
\usepackage[pdfborder={0 0 0}]{hyperref}
\usepackage{algorithm}
\usepackage{algpseudocode}
%\numberwithin{equation}{section}
%\numberwithin{figure}{section}
%\numberwithin{algorithm}{section}
\hypersetup{pdfborder=0 0 0}
%\usepackage{flafter}
%\usepackage[section]{placeins}
\newtheorem{law}{정리}
\newtheorem{theorem}{Theorem}[section]
\newtheorem{lemma}[theorem]{Lemma}
\newtheorem{proposition}[theorem]{Proposition}
\newtheorem{corollary}[theorem]{Corollary}
\newenvironment{proof1}[1][Proof]{\begin{trivlist}
\item[\hskip \labelsep {\bfseries #1}]}{\end{trivlist}}
\newenvironment{definition}[1][Definition]{\begin{trivlist}
\item[\hskip \labelsep {\bfseries #1}]}{\end{trivlist}}
\newenvironment{example}[1][Example]{\begin{trivlist}
\item[\hskip \labelsep {\bfseries #1}]}{\end{trivlist}}
\newenvironment{remark}[1][Remark]{\begin{trivlist}
\item[\hskip \labelsep {\bfseries #1}]}{\end{trivlist}}

\newcommand{\qedd}{\nobreak \ifvmode \relax \else
      \ifdim\lastskip<1.5em \hskip-\lastskip
      \hskip1.5em plus0em minus0.5em \fi \nobreak
      \vrule height0.75em width0.5em depth0.25em\fi}
\theoremstyle{examplestyle}
\newcommand{\paih}[1]{%
\index{packages!#1@\textsf{#1}}%
\index{#1@\textsf{#1}}}
\newcommand{\pai}[1]{%
\paih{#1}\textsf{#1}}
\usepackage{array}
\makeatletter
\newcolumntype{e}[1]{%--- Enumerated cells ---
   >{\minipage[t]{\linewidth}%
     \NoHyper%                Hyperref adds a vertical space
     \let\\\tabularnewline
     \enumerate
        \addtolength{\rightskip}{0pt plus 50pt}% for raggedright
        \setlength{\itemsep}{-\parsep}}%
   p{#1}%
   <{\@finalstrut\@arstrutbox\endenumerate
     \endNoHyper
     \endminipage}}

\newcolumntype{i}[1]{%--- Itemized cells ---
   >{\minipage[t]{\linewidth}%
        \let\\\tabularnewline
        \itemize
           \addtolength{\rightskip}{0pt plus 50pt}%
           \setlength{\itemsep}{-\parsep}}%
   p{#1}%
   <{\@finalstrut\@arstrutbox\enditemize\endminipage}}

\AtBeginDocument{%
    \@ifpackageloaded{hyperref}{}%
        {\let\NoHyper\relax\let\endNoHyper\relax}}
\makeatother
\setmainfont[
    Ligatures=TeX,
    Extension=.otf,
    UprightFont= *-regular,
    BoldFont=*-bold,
    ItalicFont=*-italic,
    BoldItalicFont=*-bolditalic
]{texgyreschola}
%\setmainfont[Mapping=tex-text]{TeX Gyre Pagella}
%\setsansfont[Mapping=tex-text]{Helvetica}
\setmainhangulfont[BoldFont=렉시새봄R,ItalicFont=렉시새봄R,
    ItalicFeatures={FakeSlant=.167}]{렉시새봄R}
%\setmainhangulfont[BoldFont=나눔명조 ExtraBold,ItalicFont=나눔명조,     ItalicFeatures={FakeSlant=.167}]{나눔명조}
\setsanshangulfont[BoldFont=나눔고딕 ExtraBold,ItalicFont=나눔고딕,
    ItalicFeatures={FakeSlant=.167}]{나눔고딕}
%\setmainhanjafont{네이버사전}
\makeatletter
\renewcommand{\tableofcontents}[1][\contentsname]{%
  \section*{#1}
  \begin{multicols}{2}
    \@starttoc{toc}
  \end{multicols}
}
\makeatother

\title{공학 수치해석 중간고사}
\author{}

% Change to the current month of the seriest
%\reportmonth{}
% Change to the current year of the series
%\reportyear{}
% Change to the TR number that you obtained from the
% UWEETR web pages when you initially created a new
% TR number. Only provide the last 4 digits here, the year
% goes in the \reportyear{} field above.
%\reportnumber{MIDTERM EXAM}

\begin{document}
%\renewcommand{\thelstlisting}{\thesection-\arabic{lstlisting}}
% This first line makes the cover page, which prints the TR number.
%\makecover
% This second line makes the title portion of the first page.
%\maketitle
%\tableofcontents[Table of Contents]
\begin{center}
{\lineskip .75em\begin{tabular}[t]{c}\LARGE{공학수치해석 기말고사} \large{2012.12.18}\end{tabular}\par}%
\end{center}

\begin{itemize}
\item[문제1] 다음 Figure~\ref{fig:prob1}과 같이 한쪽 끝은 열려 있으며, 두께를 무시할 수 있는 벽을 갖는 최적의 원통 용기를 설계하라. 용기는 $0.2m^{3}$의 체적을 담으려 한다. 밑면 면적과 옆면 면적이 최소화 되도록 설계하라.
\end{itemize}
\begin{figure}[!hbpt]
\centering
\includegraphics[keepaspectratio=true,width=0.4\linewidth]{finalterm/prob1.eps}
\caption{뚜껑이 없는 원통형의 용기}
\label{fig:prob1}
\end{figure}
\begin{itemize}
\item[(a)] 일반적인 최적화 문제의 수식모델을 세워라. [10점]
\item[(b)] Lagrangian법으로 설계하라(각 설계값은 과학적표기법 소숫점 4째 짜리까지 표기). [20점]
\end{itemize}
\begin{itemize}
\item[문제2] 일반강도 콘크리트 공시체 20개의 압축강도(MPa)를 측정한 결과 다음과 같은 데이터가 산출되었다. 물음에 답하라.
\end{itemize}
\begin{table}[!hbt]
\centering
\begin{tabular}{llll}
\hline\hline
21.867&19.672&24.612&22.785\\
23.736&22.215&21.933&22.583\\
21.430&21.375&23.320&20.069\\
21.548&22.505&21.409&19.450\\
19.992&20.411&21.506&20.103\\
\hline\hline
\end{tabular}
\end{table}
\begin{itemize}
\item[(a)] 평균($\bar{y}$), 표준편차($s_{y}$), 분산($s_{y}^{2}$), 분산계수(c.v)를 구하여라. (소숫점 3째 자리까지 표기) [10점]
\item[(b)] 데이터의 분포가 정규분포를 따른다고 가정하고 위에서 계산한 표준편차가 유효표준편차라고 가정하여 95\%에 포함되는 영역(즉, 하한값과 상한값)을 계산하라. (소숫점 3째 자리까지 표기) [10점]
\end{itemize}
\begin{itemize}
\item[문제3] 다음 Figure~\ref{fig:prob2}와 같이 하중을 받고 있는 단순지지된 보가 있다. 이러한 보의 전단력의 함수는 다음 식과 같이 특이함수(singularity function)을 사용하여 나타낼 수 있다.
\end{itemize}

\begin{align*}
V(x)&=20\left[\left<x-0\right>^{1}-\left<x-5\right>^{1}\right]-15\left<x-8\right>^{0}-57\\
\left<x-a\right>^{n}&=\begin{Bmatrix}\left<x-a\right>^{n}&\text{when  } x>a\\0&\text{when  } x\leq a\end{Bmatrix}
\end{align*}
\begin{figure}[!hbpt]
\centering
\includegraphics[keepaspectratio=true,width=0.7\linewidth]{finalterm/prob2.eps}
\caption{Simple supported beam}
\label{fig:prob2}
\end{figure}
전단력이 0이 되는 점들을 이분법을 써서 구하라. [30점]
\begin{algorithm}\label{alg:c5}
Choose lower $x_{l}$ and upper $x_{u}$ guesses for the root such that the function changes sign over the interval. This can be checked by ensuring that $f(x_{l})f(x_{u})<0$.
\begin{algorithmic}
\While{$f(x_{r})$ is not sufficiently small}
  \State $x_{r}=\frac{x_{l}+x_{u}}{2}$
  \If{$f(x_{l})f(x_{r})<0$}
    \State $x_{u}=x_{r}$
  \Else
    \State $x_{l}=x_{r}$
  \EndIf  
  \If{$\epsilon_{a}=\left|\frac{x_{new}-x_{old}}{x_{new}}\right|$ is sufficiently small}
    \State $x_{r}=x_{new}$
    \State \Return $x_{r}$
  \EndIf
\EndWhile
\end{algorithmic}
\caption{이분법(Bisection Method)}
\end{algorithm}
\begin{itemize}
\item[문제4] 다음 표로 주어진 데이터에 대하여 다음 문제를 풀어라
\end{itemize}
\begin{table}[!hbt]
\centering
\begin{tabular}{l|lllllllllll}
\hline\hline
x&0&1&2&3&4&5&6&7&8&9&10\\
\hline
y&-10.41&-4.03&-10.00&-0.17&5.12&14.05&19.36&34.01&55.10&94.54&96.47\\
\hline\hline
\end{tabular}
\end{table}
\begin{itemize}
\item[(a)] 직선으로 최소제곱회귀분석을 하여라. [10점]
\item[(b)] 2차 다항식으로 최소제곱회귀분석을 하여라. [10점]
\end{itemize}
%
%\begin{itemize}
%\item[문제2] 다음 Figure\ref{fig:e2}는 등분포하중을 받는 캔틸레버보를 나타낸다. 탄성곡선의 방정식은 식(\ref{eq:e2})와 같다. 다음 문항에 답하여라.
%\end{itemize}
%\begin{figure}[!hbpt]
%\centering
%\includegraphics[keepaspectratio=true,width=0.6\linewidth]{midterm/2.eps}
%\caption{Cantilever beam}
%\label{fig:e2}
%\end{figure}
%
%\begin{equation}\label{eq:e2}
%y=\frac{w_{0}}{24EI}\left(-x^4+4Lx^3-6L^2 x^2\right)
%\end{equation}
%\begin{itemize}
%\item[(a)] 캔틸레버보의 $x=50cm$일 때를 기준점으로 하여 0차에서 3차까지의 Taylor급수전개를 사용하여 $x=100cm$지점의 처짐($y$)의 근사값을 구하고, 참백분율 상대오차 $\epsilon_{t}$를 구하라. (단, 매개변수는 $L=300cm$, $E=50,000kN/cm^2$, $I=30,000cm^4$, $w_{0}=2.5kN/cm$과 같으며 참상대오차의 절단오차는 소수점 4째 자리의 과학적 표기법으로 표시한다.) [10점]
%\item[(b)] (a)의 매개변수에서 $L=300\pm5cm$ 그리고 $I=30,000\pm100cm^4$의 측정오차가 있었다. 1차 오차해석으로 $x=100cm$ 지점에서의 처짐각($dy/dx$)의 추정오차값을 구하여라. (단, 절단오차는 소수점 4째 자리의 과학적 표기법으로 표시한다.) [10점]
%\item[(c)] 처짐 $y$가 $1cm$가 되는 지점을 이분법을 사용하여 근을 구하라, $x_{l}=200cm$, $x_{u}=250cm$을 초기구간으로 가정하고, 근사오차 $\epsilon_{a}$가 1\% 이하로 떨어질 때까지 반복하라. [10점] (단, 반복횟수만큼의 열을 가지는 테이블을 작성하고 함수값등의 절단오차는 소수점 4째 자리까지 표시한다.)
%\end{itemize}
%
%\begin{itemize}
%\item[문제3] 점성감쇠조화진동(harmonic vibration with viscous damping)하는 물체가 공진상태까지 도달할 때, 정적변위응답$u_{st}$에 대한 동적변위응답$u(t)$는 $\xi$가 작은 경우 다음 근사식(\ref{eq:e4})과 같이 주어진다. 다음 문항에 답하여라.
%\end{itemize}
%\begin{align}
%%u(t)&=u_{st}\frac{1}{2\xi}\left[e^{-\xi\omega_{n}t}\left(\cos\omega_{D}t+\frac{\xi}{\sqrt{1-\xi^2}}\sin\omega_{D}t\right)-\cos\omega_{n}t\right]\label{eq:e3}\\
%u(t)&\cong u_{st}\frac{1}{2\xi}\left(e^{-\xi\omega_{n}t}-1\right)\cos\omega_{n}t\label{eq:e4}
%\end{align}
%여기서, $\xi$는 감쇠비, $\omega_{n}$은 고유진동수이다.
%\begin{itemize}
%\item[(a)] $\omega_{n}=1$이고, $\xi=0.05$일 때, 식(\ref{eq:e4})을 통해 최대변위증폭비 $\max\{u(t)/u_{st}\}$가 5에 도달하는 시간을 Newton-Raphson법을 통해 구하여라. 초기가정 $x_{0}=10$으로 세번 반복한다. [15점]
%\item[(b)] (a)를 할선법(secant method)를 사용하여 구하여라. 초기가정 $x_{0}=10$, $x_{1}=11$으로 시작하고 세번 반복하라 [15점]
%\item[(c)] (a)를 수정된 할선법(modified secant method)를 사용하여 구하여라. 초기가정 $x_{0}=10$, $\delta=0.01$로 시작하고 세번 반복하라 [15점]
%\item[$\blacktriangleright$]유의사항 : 식(\ref{eq:e4})의 1차도함수를 구하기 어려운 경우, 최대변위증폭비를 구하는 문제이기 때문에 $\max(\cos\omega_{n}t)=1$로 가정한 포락곡선함수(envelope function)를 함수로 사용하여도 되며, 증폭비는 절대값이기 때문에 함수의 근의 존재유무에 유의하라.
%\end{itemize}
%
%\clearpage
%%\subsubsection{알고리즘 테이블}
%
%\begin{algorithm}\label{alg:c5}
%Choose lower $x_{l}$ and upper $x_{u}$ guesses for the root such that the function changes sign over the interval. This can be checked by ensuring that $f(x_{l})f(x_{u})<0$.
%\begin{algorithmic}
%\While{$f(x_{r})$ is not sufficiently small}
%  \State $x_{r}=\frac{x_{l}+x_{u}}{2}$
%  \If{$f(x_{l})f(x_{r})<0$}
%    \State $x_{u}=x_{r}$
%  \Else
%    \State $x_{l}=x_{r}$
%  \EndIf  
%  \If{$\epsilon_{a}=\left|\frac{x_{new}-x_{old}}{x_{new}}\right|$ is sufficiently small}
%    \State $x_{r}=x_{new}$
%    \State \Return $x_{r}$
%  \EndIf
%\EndWhile
%\end{algorithmic}
%\caption{이분법(Bisection Method)}
%\end{algorithm}
%
%\begin{algorithm}\label{alg:c1}
%Choose an initial guess $x_{0}$.
%\begin{algorithmic}
%\For{$i=0,1,2,\cdots$}
%  \If{$f(x_{i})$ is sufficiently small}
%    \State $x_{r}=x_{i}$
%    \State \Return $x_{r}$
%  \EndIf
%  \State $x_{i+1}=x_{i}-\frac{f(x_{i})}{f'(x_{i})}$
%  \If{$\left|x_{i+1}-x_{i}\right|$ is sufficiently small}
%    \State $x_{r}=x_{i+1}$
%    \State \Return $x_{r}$
%  \EndIf
%\EndFor
%\end{algorithmic}
%\caption{Newton-Raphson Method}
%\end{algorithm}
%
%\begin{algorithm}\label{alg:c2}
%Choose an initial guess $x_{0}$ and $x_{1}$.
%\begin{algorithmic}
%\For{$i=0,1,2,\cdots$}
%  \If{$f(x_{i})$ is sufficiently small}
%    \State $x_{r}=x_{i}$
%    \State \Return $x_{r}$
%  \EndIf
%  \State $x_{i+2}=x_{i+1}-\frac{f(x_{i+1})(x_{i}-x_{i+1})}{f(x_{i})-f(x_{i+1})}$
%  \If{$\left|x_{i+2}-x_{i+1}\right|$ is sufficiently small}
%    \State $x_{r}=x_{i+2}$
%    \State \Return $x_{r}$
%  \EndIf
%\EndFor
%\end{algorithmic}
%\caption{Secant Method}
%\end{algorithm}
%
%\begin{algorithm}\label{alg:c3}
%Choose an initial guess $x_{0}$ and $\delta$.
%\begin{algorithmic}
%\For{$i=0,1,2,\cdots$}
%  \If{$f(x_{i})$ is sufficiently small}
%    \State $x_{r}=x_{i}$
%    \State \Return $x_{r}$
%  \EndIf
%  \State $x_{i+1}=x_{i}-\frac{\delta x_{i}f(x_{i})}{f(x_{i}+\delta x_{i})-f(x_{i})}$
%  \If{$\left|x_{i+1}-x_{i}\right|$ is sufficiently small}
%    \State $x_{r}=x_{i+1}$
%    \State \Return $x_{r}$
%  \EndIf
%\EndFor
%\end{algorithmic}
%\caption{Modified Secant Method}
%\end{algorithm}
\end{document}
