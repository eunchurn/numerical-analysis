% !TEX TS-program = xelatex
% !TEX encoding = UTF-8
\documentclass[a4,10pt]{article}
\usepackage{exam}
%\usepackage{pslatex}
\usepackage{graphicx}
\usepackage[toc,page]{appendix}
\usepackage[framed,numbered,autolinebreaks,useliterate]{mcode}
\input{listingstyle}
\renewcommand*{\lstlistingname}{Code [\textsc{\scriptsize{MATLAB}}]}
\usepackage{mathptmx}	% Use the Postscript Times font
\usepackage[FIGBOTCAP,normal,bf,tight]{subfigure}
\usepackage[normal,bf,tight]{subfigure}
\usepackage[dvips,light,first,bottomafter]{draftcopy}
\draftcopyName{Sample, contains no OUO}{70}
\usepackage{amsmath,amssymb,amsthm}
\usepackage{kotex}
\usepackage{multirow}
\usepackage{multicol}
\usepackage[pdfborder={0 0 0}]{hyperref}
\usepackage{algorithm}
\usepackage{algpseudocode}
%\numberwithin{equation}{section}
%\numberwithin{figure}{section}
\numberwithin{algorithm}{section}
\hypersetup{pdfborder=0 0 0}
%\usepackage{flafter}
%\usepackage[section]{placeins}
\newtheorem{law}{정리}
\newtheorem{theorem}{Theorem}[section]
\newtheorem{lemma}[theorem]{Lemma}
\newtheorem{proposition}[theorem]{Proposition}
\newtheorem{corollary}[theorem]{Corollary}
\newenvironment{proof1}[1][Proof]{\begin{trivlist}
\item[\hskip \labelsep {\bfseries #1}]}{\end{trivlist}}
\newenvironment{definition}[1][Definition]{\begin{trivlist}
\item[\hskip \labelsep {\bfseries #1}]}{\end{trivlist}}
\newenvironment{example}[1][Example]{\begin{trivlist}
\item[\hskip \labelsep {\bfseries #1}]}{\end{trivlist}}
\newenvironment{remark}[1][Remark]{\begin{trivlist}
\item[\hskip \labelsep {\bfseries #1}]}{\end{trivlist}}

\newcommand{\qedd}{\nobreak \ifvmode \relax \else
      \ifdim\lastskip<1.5em \hskip-\lastskip
      \hskip1.5em plus0em minus0.5em \fi \nobreak
      \vrule height0.75em width0.5em depth0.25em\fi}
\theoremstyle{examplestyle}
\newcommand{\paih}[1]{%
\index{packages!#1@\textsf{#1}}%
\index{#1@\textsf{#1}}}
\newcommand{\pai}[1]{%
\paih{#1}\textsf{#1}}
\usepackage{array}
\makeatletter
\newcolumntype{e}[1]{%--- Enumerated cells ---
   >{\minipage[t]{\linewidth}%
     \NoHyper%                Hyperref adds a vertical space
     \let\\\tabularnewline
     \enumerate
        \addtolength{\rightskip}{0pt plus 50pt}% for raggedright
        \setlength{\itemsep}{-\parsep}}%
   p{#1}%
   <{\@finalstrut\@arstrutbox\endenumerate
     \endNoHyper
     \endminipage}}

\newcolumntype{i}[1]{%--- Itemized cells ---
   >{\minipage[t]{\linewidth}%
        \let\\\tabularnewline
        \itemize
           \addtolength{\rightskip}{0pt plus 50pt}%
           \setlength{\itemsep}{-\parsep}}%
   p{#1}%
   <{\@finalstrut\@arstrutbox\enditemize\endminipage}}

\AtBeginDocument{%
    \@ifpackageloaded{hyperref}{}%
        {\let\NoHyper\relax\let\endNoHyper\relax}}
\makeatother
\setmainfont[
    Ligatures=TeX,
    Extension=.otf,
    UprightFont= *-regular,
    BoldFont=*-bold,
    ItalicFont=*-italic,
    BoldItalicFont=*-bolditalic
]{texgyreschola}
%\setmainfont[Mapping=tex-text]{TeX Gyre Pagella}
%\setsansfont[Mapping=tex-text]{Helvetica}
\setmainhangulfont[BoldFont=렉시새봄R,ItalicFont=렉시새봄R,
    ItalicFeatures={FakeSlant=.167}]{렉시새봄R}
%\setmainhangulfont[BoldFont=나눔명조 ExtraBold,ItalicFont=나눔명조,     ItalicFeatures={FakeSlant=.167}]{나눔명조}
\setsanshangulfont[BoldFont=나눔고딕 ExtraBold,ItalicFont=나눔고딕,
    ItalicFeatures={FakeSlant=.167}]{나눔고딕}
%\setmainhanjafont{네이버사전}
\makeatletter
\renewcommand{\tableofcontents}[1][\contentsname]{%
  \section*{#1}
  \begin{multicols}{2}
    \@starttoc{toc}
  \end{multicols}
}
\makeatother

\title{공학 수치해석 중간고사}
\author{}

% Change to the current month of the seriest
%\reportmonth{}
% Change to the current year of the series
%\reportyear{}
% Change to the TR number that you obtained from the
% UWEETR web pages when you initially created a new
% TR number. Only provide the last 4 digits here, the year
% goes in the \reportyear{} field above.
%\reportnumber{MIDTERM EXAM}

\begin{document}
%\renewcommand{\thelstlisting}{\thesection-\arabic{lstlisting}}
% This first line makes the cover page, which prints the TR number.
%\makecover
% This second line makes the title portion of the first page.
%\maketitle
%\tableofcontents[Table of Contents]
\begin{center}
{\lineskip .75em\begin{tabular}[t]{c}\LARGE{공학수치해석 중간고사} \large{2012.11.1}\end{tabular}\par}%
\end{center}

\begin{itemize}
\item[문제1] 다음 Figure~\ref{fig:e1}은 전투기가 착륙할 때 펼쳐지는 drag parachute 를 보여준다. 이 때 전투기는 속도함수의 가속도($a=-0.004v^2 m/sec^2$)의 지배를 받게 된다. 다음 문항에 답하여라.
\end{itemize}
\begin{figure}[!hbpt]
\centering
\includegraphics[keepaspectratio=true,width=0.6\linewidth]{midterm/dragchute.eps}
\caption{Drag chute}
\label{fig:e1}
\end{figure}
\begin{itemize}
\item[(a)] 감속하는 전투기의 속도에 대한 수학적모델을 세우고, 독립변수와 종속변수를 나타내시오. [5점]
\item[solution (a)]
\begin{displaymath}
a=\frac{dv}{dt}=-0.004v^2
\end{displaymath}
종속변수:$a$,$v$, 독립변수:$t$
\item[(b)] 감속하는 전투기의 속도 $80m/s$에서 $10m/s$에 도달하기 까지 걸리는 시간의 해석해(exact solution)를 구하시오. [10점]
\item[solution (b)]
\begin{align*}
\frac{dv}{v^2}&=0.004dt\\
\int_{80}^{v}\frac{1}{v^2}dv&=\int_{0}^{t}-0.004dt\\
\left[\frac{-1}{v}\right]_{80}^{v}&=-0.004t\\
\left(\frac{1}{v}-\frac{1}{80}\right)&=0.004t\\
t&=250\left(\frac{1}{v}-\frac{1}{80}\right)\\
\therefore t(v=10)&=21.9~(second)
\end{align*}
\item[(c)] (b)에서 동일하게 감속하는 시간동안 이동한 거리의 해석해(exact solution) 혹은 Euler법을 사용한 수치해(numerical solution)을 구하시오. (단, 수치해를 구할때 독립변수의 간격은 2로 하고, 각 단계에서의 절단오차 소숫점 4째자리까지 나타내시오) [10점]
\item[solution (c)]
\item[exact sol.]
\begin{align*}
a&=\frac{dv}{dt}\\
\frac{dv}{ds}\frac{ds}{dt}&=\frac{dv}{ds}v\\
\frac{dv}{v}&-0.004ds\\
\int_{80}^{v}\frac{dv}{v}&=\int_{0}^{s}-0.004ds\\
\left[\ln v\right]_{80}^{v}&=-0.004s\\
\ln v -\ln 80&=-0.004s\\
s&=250\ln\left(\frac{80}{v}\right)\\
\therefore s(v=10)&=520~(m)
\end{align*}
\item[num. sol.]
\begin{align*}
\left(\frac{1}{v}-\frac{1}{80}\right)&=0.004t\\
v&=\left(0.004t+\frac{1}{80}\right)^{-1}\\
\frac{ds}{dt}&=\left(0.004t+\frac{1}{80}\right)^{-1}\\
s_{i+1}&=s_{i}+(t_{i+1}-t_{i}) \left(0.004t_{i}+\frac{1}{80}\right)^{-1}
\end{align*}
\begin{table}[!hbpt]
\centering
\begin{tabular}{c|c}
\hline\hline
Time(second)&Distance(m)\\
\hline
0&0\\
2&160\\
4&257.561\\
6&327.7364\\
8&382.5309\\
10&427.4748\\
12&465.57\\
14&498.6278\\
16&527.8249\\
18&553.9687\\
20&577.6374\\
22&599.259\\
\hline\hline
\end{tabular}
\end{table}

\end{itemize}

\begin{itemize}
\item[문제2] 다음 Figure\ref{fig:e2}는 등분포하중을 받는 캔틸레버보를 나타낸다. 탄성곡선의 방정식은 식(\ref{eq:e2})와 같다. 다음 문항에 답하여라.
\end{itemize}
\begin{figure}[!hbpt]
\centering
\includegraphics[keepaspectratio=true,width=0.6\linewidth]{midterm/2.eps}
\caption{Cantilever beam}
\label{fig:e2}
\end{figure}

\begin{equation}\label{eq:e2}
y=\frac{w_{0}}{24EI}\left(-x^4+4Lx^3-6L^2 x^2\right)
\end{equation}
\begin{itemize}
\item[(a)] 캔틸레버보의 $x=50cm$일 때를 기준점으로 하여 0차에서 3차까지의 Taylor급수전개를 사용하여 $x=100cm$지점의 처짐($y$)의 근사값을 구하고, 참백분율 상대오차 $\epsilon_{t}$를 구하라. (단, 매개변수는 $L=300cm$, $E=50,000kN/cm^2$, $I=30,000cm^4$, $w_{0}=2.5kN/cm$과 같다.) [10점]
\item[solution (a)]
참값 $f(100)=-2.9861 \times 10^{-1}$
\begin{table}[!hbpt]
\centering
\begin{tabular}{c|c|c|c}
\hline\hline
Order&Equation&Result&$\epsilon_{t}$\\
\hline
0&$f(100)\cong f(50)$&$f(100)\cong -8.3767\times 10^{-2}$& 71.9477\%\\
1&$f(100)\cong f(50)+f'(50)(50)$&$f(100)\cong -2.4175\times 10^{-1}$ &  19.0407\%\\
2&$f(100)\cong f(50)+f'(50)(50)+\frac{1}{2!}f''(50)(50)^2$&$f(100)\cong-3.0686\times 10^{-1}$&2.7616\%\\
3&$f(100)\cong f(50)+f'(50)(50)+\frac{1}{2!}f''(50)(50)^2+\frac{1}{3!}f'''(50)(50)^3$&$f(100)\cong = -2.9818\times 10^{-1}$&0.1453\%
\\ \hline\hline
\end{tabular}
\end{table}

%\item[zero order] $f(100)\cong f(50)=-8.3767\times 10^{-2}$
%\item[first order] $f(100)\cong f(50)+f'(50)(50)=-2.4175\times 10^{-1}$
%\item[second order] $f(100)\cong f(50)+f'(50)(50)+\frac{1}{2!}f''(50)(50)^2=-3.0686\times 10^{-1}$
%\item[third order] $f(100)\cong f(50)+f'(50)(50)+\frac{1}{2!}f''(50)(50)^2+\frac{1}{3!}f'''(50)(50)^3= -2.9818\times 10^{-1}$

\item[(b)] (a)의 매개변수에서 $L=300\pm5cm$ 그리고 $I=30,000\pm100cm^4$의 측정오차가 있었다. 1차 오차해석으로 $x=100cm$ 지점에서의 처짐각($dy/dx$)의 추정오차값을 구하여라. (단, 절단오차는 소수점 4째 자리의 과학적 표기법으로 표시한다.) [10점]
\item[solution (b)]
\begin{align*}
\Delta y'&=\left|\frac{\partial y'}{\partial L}\right|\Delta L+\left|\frac{\partial y'}{\partial I}\right|\Delta I\\
&=\left|\frac{w_{0}}{24EI}(12x^2-24Lx)\right|\Delta L + \left|-\frac{w_{0}}{24EI^2}(-4x^3+12Lx^2-12L^2 x)\right|\Delta I\\
\therefore \frac{dy}{dx}&=-5.2778 \times 10^{-3} \pm 2.2593 \times 10^{-4}
\end{align*}
\item[(c)] 처짐 $y$가 $1cm$가 되는 지점을 이분법을 사용하여 근을 구하라, $x_{l}=200cm$, $x_{u}=250cm$을 초기구간으로 가정하고, 근사오차 $\epsilon_{a}$가 1\% 이하로 떨어질 때까지 반복하라. [10점] (단, 반복횟수만큼의 열을 가지는 테이블을 작성하고 함수값등의 절단오차는 소수점 4째 자리의 과학적 표기법으로 표시한다.)
\item[solution (c)]
\begin{table}[!hbpt]
\centering
\begin{tabular}{c|c|c|c|c|c|c}
\hline\hline
Iter.&$x_l$&$x_u$&$x_r$&$f'(x_r)$&$f(x_r)$&$\epsilon_{a}$\\
\hline
1&200&250&225&-0.00076538&-0.1272&11.1111\%\\
2&200&225&212.5&-0.00097819&-0.035321&5.8824\%\\
3&200&212.5&206.25&-0.0010825&0.010261&3.0303\%\\
4&206.25&212.5&209.375&-0.0010305&-0.012497&1.4925\%\\
5&206.25&209.375&207.8125&-0.0010566&-0.0011094&0.75188\%\\
\hline\hline
\end{tabular}
\end{table}

\end{itemize}

\begin{itemize}
\item[문제3] 점성감쇠조화진동(harmonic vibration with viscoud damping)하는 물체가 공진상태까지 도달할 때, 정적변위응답$u_{st}$에 대한 동적변위응답$u(t)$는 $\xi$가 작은 경우 다음 근사식(\ref{eq:e4})과 같이 주어진다. 다음 문항에 답하여라.
\end{itemize}
\begin{align}
%u(t)&=u_{st}\frac{1}{2\xi}\left[e^{-\xi\omega_{n}t}\left(\cos\omega_{D}t+\frac{\xi}{\sqrt{1-\xi^2}}\sin\omega_{D}t\right)-\cos\omega_{n}t\right]\label{eq:e3}\\
u(t)&\cong u_{st}\frac{1}{2\xi}\left(e^{-\xi\omega_{n}t}-1\right)\cos\omega_{n}t\label{eq:e4}
\end{align}
여기서, $\xi$는 감쇠비, $\omega_{n}$은 고유진동수이다.
\begin{itemize}
\item[(a)] $\omega_{n}=1$이고, $\xi=0.05$일 때, 식(\ref{eq:e4})을 통해 최대변위증폭비 $\max\{u(t)/u_{st}\}$가 5에 도달하는 시간을 Newton-Raphson법을 통해 구하여라. 초기가정 $x_{0}=10$으로 세번 반복한다. [15점]
\item[(b)] (a)를 할선법(secant method)를 사용하여 구하여라. 초기가정 $x_{0}=10$, $x_{1}=11$으로 시작하고 세번 반복하라 [15점]
\item[(c)] (a)를 수정된 할선법(modified secant method)를 사용하여 구하여라. 초기가정 $x_{0}=10$, $\delta=0.01$로 시작하고 세번 반복하라 [15점]
\item[$\blacktriangleright$]유의사항 : 식(\ref{eq:e4})의 1차도함수를 구하기 어려운 경우, 최대변위증폭비를 구하는 문제이기 때문에 $\max(\cos\omega_{n}t)=1$로 가정한 포락곡선함수(envelope function)를 함수로 사용하여도 되며, 증폭비는 절대값이기 때문에 함수의 근의 존재유무에 유의하라.
\end{itemize}

\clearpage
\begin{itemize}
\item[문제3(CASE-1)] 식(\ref{eq:e4})을 사용하는 경우 Newton-Raphson법에서는 1차도함수$f'(t)$를 구해야한다.
\begin{equation}
f'(t)=-\frac{\omega_{n}}{2\xi}\left\{\xi e^{-\xi\omega_{n}t}\cos\omega_{n}t+\left(e^{-\xi\omega_{n}t}-1\right)\sin\omega_{n}t\right\}
\end{equation}
\item[(a)] Newton-Raphson method
\begin{table}[!hbpt]
\centering
\begin{tabular}{c|c|c|c|c}
\hline\hline
Iter.&$x_{i}$&$f(x_{i})$&$f'(x_{i})$&$\epsilon_{a}$\\
\hline
0&10&1.0653&-1.8861&5.3462\%\\
1&10.5648&0.89641&-3.6056&2.2991\%\\
2&10.8134&0.82357&-4.0546&1.8438\%\\
\hline\hline
\end{tabular}
\end{table}

\item[(b)] Secant method
\begin{table}[!hbpt]
\centering
\begin{tabular}{c|c|c|c|c|c}
\hline\hline
Iter.&$x_{i}$&$x_{i+1}$&$f(x_{i})$&$f(x_{i+1})$&$\epsilon_{a}$\\
\hline
0&10&11&1.0653&0.7695&19.1256\%\\
1&11&13.6013&0.7695&0.065831&1.7578\%\\
2&13.6013&13.8447&0.065831&0.0045619&0.13071\%\\
\hline\hline
\end{tabular}
\end{table}

\item[(c)] Modified secant method
\begin{table}[!hbpt]
\centering
\begin{tabular}{c|c|c|c|c}
\hline\hline
Iter.&$x_{i}$&$f(x_{i})$&$f(x_{i}+\delta x_{i})$&$\epsilon_{a}$\\
\hline
0&10&1.0653&1.0351&26.0441\%\\
1&13.5216&0.086074&0.051804&2.4501\%\\
2&13.8612&0.00043785&-0.034098&0.012676\%\\
\hline\hline
\end{tabular}
\end{table}
\end{itemize}

\clearpage
\begin{itemize}
\item[문제3(CASE-2)] 포락곡선함수를 사용하는 경우 포락곡선함수는 다음 식(\ref{eq:a3-1})과 같다. 0에서 작아지는 함수이므로, $f(t)+5=0$으로 함수를 구성해야한다.
\begin{align}
\frac{u(t)}{u_{st}}=f(t)&=\frac{1}{2\xi}\left(e^{-\xi\omega_{n}t}-1\right)\label{eq:a3-1}\\
f'(t)&=-\frac{\omega_{n}}{2}e^{-\xi\omega_{n}t}
\end{align}
\item[(a)] Newton-Raphson method
\begin{table}[!hbpt]
\centering
\begin{tabular}{c|c|c|c|c}
\hline\hline
Iter.&$x_{i}$&$f(x_{i})$&$f'(x_{i})$&$\epsilon_{a}$\\
\hline
0&10&1.0653&-0.30327&25.996\%\\
1&13.5128&0.08831&-0.25442&2.5044\%\\
2&13.8599&0.00076191&-0.25004&0.021981\%\\
\hline\hline
\end{tabular}
\end{table}

\item[(b)] Secant method
\begin{table}[!hbpt]
\centering
\begin{tabular}{c|c|c|c|c|c}
\hline\hline
Iter.&$x_{i}$&$x_{i+1}$&$f(x_{i})$&$f(x_{i+1})$&$\epsilon_{a}$\\
\hline
0&10&11&1.0653&0.7695&19.1256\%\\
1&11&13.6013&0.7695&0.065831&1.7578\%\\
2&13.6013&13.8447&0.065831&0.0045619&0.13071\%\\
\hline\hline
\end{tabular}
\end{table}

\item[(c)] Modified secant method
\begin{table}[!hbpt]
\centering
\begin{tabular}{c|c|c|c|c}
\hline\hline
Iter.&$x_{i}$&$f(x_{i})$&$f(x_{i}+\delta x_{i})$&$\epsilon_{a}$\\
\hline
0&10&1.0653&1.0351&26.0441\%\\
1&13.5216&0.086074&0.051804&2.4501\%\\
2&13.8612&0.00043785&-0.034098&0.012676\%\\
\hline\hline
\end{tabular}
\end{table}
\end{itemize}

\end{document}
