최적화는 \ref{sec:brack}장과 \ref{sec:open}장과 같이 둘 다 함수 위의 한 점을 추측하거나 찾는다는 점에서 관련성이 있다. 두 방법의 근본적인 차이점으로 구간법은 함수들의 근을 구하는 반면 최적화(optimization)은 최소값 혹은 최대값을 찾는 것이다.\\ 최적점은 곡선 위의 평탄한 점으로, 수학적 용어로 나타내면 도함수 $f'(x)$가 $0$이 되는 점에 해당한다. 또한 2차도함수 $f''(x)$의 부호에 의해 최적점이 최소값인지 최대값인지를 나타낸다. 하지만 $f'(x)$가 손쉽게 구해지지 않는 경우 이러한 최적화 방법은 종종 복잡해지게 된다. 이 경우 도함수를 근사하기 위하여 유한자분 근사법을 사용하여야 한다.\\최적화는 근 구하는 방법 이상으로 단순히 근 구하기가 아닌 다른 수학적 접근을 포함한다. 이러한 접근은 다차원의 최적화를 더욱 가능하게 한다.
\begin{itemize}
\item 최소 중량 및 최대 강도의 항공기 설계
\item 우주선의 최적궤도
\item 최소 비용의 토목공학 구조물 설계
\item 최대 수력에너지를 내면서 호수 때 손실을 최소화하는 댐의 수자원 프로젝트
\item 위치에너지를 최소화하는 구조물 거동의 예측
\item 최소 비용의 재료 절삭 방법
\item 최대 효율의 펌프와 열전달 장치의 설계
\item 전기회로와 기계의 열발생을 최소화하면서 최대 출력설계
\item 한 번의 판촉 여행으로 여러 도시를 방문시 가장 짧은 경로 계산
\item 최적 계획과 스케쥴링(scheduling)
\item 최소 오차를 갖는 통계분석과 모델
\item 최적 파이프라인 회로 설계
\item 재고 제어
\item 비용을 최소화하는 보수계획
\item 대기시간과 준비시간의 최소화
\item 최소 비용으로 수질표준을 만족하는 폐기물 처리 시스템 설계
\end{itemize}

지금까지 다룬 대부분의 공학적 모델들은 서술적(descriptive) 모델이었다. 즉, 공학적 장치나 시스템의 거동을 나타내기 위하여 유도된 모델이다. 반면에 최적화는 문제의 "최상의 결과" 혹은 최적해를 구하는 것을 다루는 일련의 과정, 혹은 가장 좋은 설계를 설정하게 되는 설정적(prescriptive) 모델이라는 용어를 사용한다. 즉, 엔지니어는 효율적인 방법으로 프로젝트를 수행하거나 제품을 개발하여야 하는데 그 과정중에 실제 문제들에 의한 물리적 제약이 존재하게 된다. 또는 비용을 계속적으로 줄여야만 한다. 따라서 성능과 제약조건 사이의 균형을 구하는 최적화 문제에 항상 부딪히게 된다.
\\
\framebox{예제} \textbf{낙하산 비용의 최적화}\\
\rule{\textwidth}{0.1pt}
전쟁지역의 피난민에게 물자들을 공수하는 기관에 속한 엔지니어라고 가정하자. 물자들을 가능한 한 낮은 고도에서 떨어뜨려 관측되지 않으면서 가능한 피난민 캠프에 근접하려고 한다. 낙하산은 수송기에서 떨어지자마자 펼쳐지며, 손상을 줄이기 위하여 지면에 도착하는 충격속도가 적어도 임계속도 $v_{c}=20 m/sec$보다는 작아야한다. 낙하용 낙하산의 단면적은 반구의 단면적과 같다.
\begin{equation*}
A=2\pi r^2
\end{equation*}
질량을 낙하산에 연결하는 16개의 줄이 길이는 다음 식과 같이 낙하산 반지름으로 표시된다.
\begin{equation*}
l=\sqrt{2}r
\end{equation*}
낙하산에 걸리는 항력계수는 다음 식과 같이 단면적의 선형함수로 표현된다.
\begin{equation*}
c=k_{c}A
\end{equation*}
여기서, $c$는 항력계수($kg/s$), $k_{c}$는 항력에 대한 면적의 영향을 나타내는 비례상수로 [$kg/(s\cdot m^{2})$]의 단위를 갖는다. 또한 전체 하중을 여러 개의 뭉치로 나누면, 각각의 뭉치 덩어리 질량은 다음과 같이 계산된다.
\begin{equation*}
m=\frac{M_{t}}{n}
\end{equation*}
여기서 $m$은 각 뭉치의 질량($kg$)이며, $M_{t}$는 낙하된 전체질량($kg$), 그리고 $n$의 전체 뭉치수이다. 끝으로 각 낙하산의 가격은 다음 식과 같이 비선형식으로 표시된다.
\begin{equation*}
p=c_{0}+c_{1}l+c_{2}A^{2}
\end{equation*}
여기서, $p$는 대당가격, $c_{0}$, $c_{1}$, $c_{2}$는 가격 계수들이다. $c_{0}$는 낙하산들의 기본 가격이며, 큰 낙하산은 작은 크기의 낙하산에 비해 제작하기가 훨씬 어렵기 때문에 면적에 대하여 비선형적 가격증가를 보인다.\\충분히 작은 낙하 충격속도를 가지면서 최소의 비용이 되도록 낙하산 크기 $r$과 개수 $n$을 결정하라.\\
목적함수를 정의하면 다음식으로 표현된다.
\begin{equation*}
\begin{aligned}
& \underset{n,r}{\text{minimize}}
& & C(n,r) = np\\
& \text{subject to}
& & v \leq v_{c}, \; n \geq 1, n\in\mathbb{Z}
\end{aligned}
\end{equation*}
이 문제는 비선형 구속화 문제가 된다. 넓은 의미로는 수식화 되었어도 어떻게 충격속도 $v$를 구할 것이가라는 다른 문제가 남는다. 낙하하는 물체의 속도는 다음과 같이 계산되었다.
\begin{equation}\label{eq:1-10}
v=\frac{gm}{c}\left(1-e^{-(c/m)t}\right)
\end{equation}
여기서 $v$는 속도($m/sec$), $g$는 중력가속도 ($m/sec^2$), $m$은 질량($kg$), 그리고 $t$는 시간($sec$)이다. 속도와 시간의 관계가 주어져있지만 물체가 떨어지는데 걸리는 시간 $t$를 결정해야하고, 지면에 닿을때의 속도가 임계속도 $v_c$ 이하여야 한다. 낙하위치에 따라 지면에 도달할때의 속도와 시간이 달라지기 때문에 식(\ref{eq:1-10})을 적분하여 얻어야 한다. 낙하높이 $z$와 낙하시간 $t$사이의 관계는
\begin{align}
z&=\int_{0}^{t}\frac{gm}{c}\left(1-e^{-(c/m)t}\right)dt\\
&=z_{0}-\frac{gm}{c}t+\frac{gm^2}{c^2}\left(1-e^{-(c/m)t}\right)\label{eq:pt4-9}
\end{align}
여기서 $z_{0}$는 낙하 최초의 높이($m$)다. 즉 이식을 통해 시간 $t$가 주어지면 높이 $z$를 예측할 수 있다. 그러나 이 문제는 높이 $z_{0}$만큼 떨어지는데 걸리는 시간을 계산하여야 하기 때문에 식(\ref{eq:pt4-9})의 근을 구하눈 문제로 다시 수식화 하여야 한다. 즉, 높이 $z$가 $0$이 되는데 필요한 시간에 대한 식을 쓰면,
\begin{align}
f(t)&=z_{0}-\frac{gm}{c}t+\frac{gm^2}{c^2}\left(1-e^{-(c/m)t}\right)\label{eq:pt4-10}\\
&=0\\
root\left[z_{0}-\frac{gm}{c}t+\frac{gm^2}{c^2}\left(1-e^{-(c/m)t}\right)\right]
\end{align}
으로 정리가 된다. 이때 시간을 구하기 위해 구간법(수치해법)을 사용하여 근을 구하여야 한다. 그 이후 시간$t$가 계산이 되면, 지면에 닿을때의 속도$v$를 구할 수 있다. 이 속도가 임계속도 이내인지 판별해야하는 조건이 추가가 된다.\\

다음 조건으로 총가격을 최소화 시키는 최적의 뭉치의 개수 $n$과, 최적의 낙하산의 반지름 $r$을 찾아라.

\begin{table}[!hbt]
\centering
\begin{tabular}{c|c|c|c}
\hline\hline
parameter name&parameters&value&unit\\
\hline
초기 낙하높이&$z_{0}$&100&$m$\\
면적당 항력비례상수&$k_{c}$&200&$kg/(s\cdot m)^2$\\
수송할 물자의 전체 질량전체질량&$M_{t}$&100&$ton$\\
낙하산 개당 기본 제작단가&$c_{0}$&1,000&KRW\\
낙하산 줄의 길이당 단가&$c_{1}$&300&$\text{KRW}/m$\\
낙하산 섬유 면적당 단가&$c_{2}$&50&$\text{KRW}/m^2$\\
\hline\hline
\end{tabular}
\end{table}
이 계산을 수계산 혹은 컴퓨터계산을 통해 결과를 얻어내는데 어떠한 수치해석 방법을 사용하여도 무방하다. 또한 최적의 결과(가격)가 아니어도 조건은 만족시켜야 한다.
단, 수계산의 경우 최소한 10번의 시행오차과정을 컴퓨터계산의 경우 20번의 시행오차과정을 수행하고, 최적의 단가, 개수, 전체가격, 지면에 도달할 때의 시간, 지면에 도달할 때의 속도를 테이블로 작성하라.\\
\framebox{해} 낙하산 최적화 문제는 "구속 비선형 다변수 함수 최적화(constrained nonlinear multivariable function minimization)" 방식을 사용한다. 이 함수는 구속최적화~\pageref{sec:optoolbox}페이지 \ref{sec:optoolbox}장에서 다룬다. 일반적으로 본 강의에서는 MATLAB의 \texttt{fmincon()}을 사용하도록 한다. 우선 낙하산이 지면에 도달할 때의 시간 $t$를 구하기 위한 낙하거리의 함수를 정의하자.
\lstinputlisting[language=Matlab, caption=낙하 거리 함수 \texttt{fdist(t,g,m,c,z0)}]{MATLAB/optimizechap/fdist.m} 
구간법 \texttt{fminbnd()}함수를 이용하여 근을 구하고, 도달거리의 조건에 대한 함수를 정의하자, 이 때, 구속조건은 부등식구속조건 \texttt{con}과 등식구속조건 \texttt{ceq}가 사용이 되는데, 등식구속조건은 사용되지 않기 때문에 \texttt{ceq}의 리턴값은 없다. 구속조건함수를 \texttt{pricecon(x)}로 정의하자.
\lstinputlisting[language=Matlab, caption=구속조건함수 \texttt{pricecon(x)}]{MATLAB/optimizechap/pricecon.m} 
그리고 마지막으로 목적함수인 총 낙하산 가격에 대한 함수를 정의한다.
\lstinputlisting[language=Matlab, caption=목적함수 \texttt{pricep(x)}]{MATLAB/optimizechap/pricep.m}
구속최적화 방법을 수행하는데 낙하거리의 함수는 근을 구하기 위한 함수였고 두가지함수(목적함수, 구속조건함수)만 필요로한다. 이를 통해 총 낙하산 가격의 최소화 구문은\\ \texttt{[x,fval]=fmincon(@pricep,x0,[],[],[],[],lb,ub,@pricecon,options);}를 통해 구하게 된다. 메인프로그램은 다음 코드와 같다,
\lstinputlisting[language=Matlab, caption=최적화 실행프로그램]{MATLAB/optimizechap/optexam1.m}
메인프로그램을 수행했을 때, 결과 값을 보자.
\begin{table}[!hbt]
\centering
\begin{tabular}{c|c|c}
\hline\hline
Parameters&Value&Unit\\
\hline
Number of parachutes ($n$)&49&EA\\
Radius of parachute ($r$)&0.88057&meter\\
Drag coefficient ($c$)&974.3905&$kg/sec$\\
Mass per each ($m$)&2040.8163&kg\\
Arrival time ($t$)&6.8883&$sec$\\
Arrival velocity ($v$)&19.7601&$m/sec$\\
Unit price ($p$)&2,561&KRW/EA\\
Total price ($C$)&125,460&KRW\\
\hline\hline
\end{tabular}
\end{table}
이 문제에는 공학적인 접근에서 일반적으로 접하게 되는 최적화 문제들의 기초적인 요소들을 대부분 포함하고 있다. 이러한 최적화 문제들은
\begin{itemize}
\item 우리의 목표를 표현하는 "목적함수"를 포함한다.
\item 많은 "설계변수"를 포함하며, 실수 혹은 정수가 될 수도 있다. 낙하산 예제에서 $r$은 실수이며, $n$은 정수이다.
\item 설계에 대한 제약을 반영하는 "구속조건"을 포함한다.
\end{itemize}
위의 과제와 예제를 통해 우리는 최적화를 수행하는데 필요한 함수를 공부할 것이다. 컴퓨터 프로그램을 수행하는데 있어서, 논리적으로 최적화 문제의 해를 구하는 것도 중요하지만 이론적인 배경도 함께 숙지하고 있어야한다. 도구에 종속되어 공학적인 문제들을 수행하게 되면 그 컴퓨터 연산결과에 대하여 신하게 되며 공학적인 고찰또한 결여되게 된다. 가능한 해 집단에서 최종적으로 우리는 특정한 해를 사용하게 되기 때문에, 이러한 고찰이 매우 중요하다.

최적화(optimization) 문제는 일반적으로 다음과 같이 표현된다.\\
$f(x)$를 최소화 혹은 최대화하며 다음의 조건을 만족하는 변수 $x$를 구하라.
\begin{align}
d_{i}(\mathbf{x})&\leq a_{i}&i&=1,2,\cdots,m\label{eq:pt4-20}\\
e_{i}(\mathbf{x})&=b_{i}&i&=1,2,\cdots,p\label{eq:pt4-21}
\end{align}
여기서 $\mathbf{x}$는 $n$차원의 설계벡터(design vector)이고, $f(\mathbf{x})$는 목적함수이며, $d_{i}(\mathbf{x})$는 부등식구속조건(inequality constraints), $e_{i}(\mathbf{x})$는 등식구속조건(equality constraints), 그리고 $a_{i}$와 $b_{i}$는 상수들이다.\\
최적화 문제들은 $f(\mathbf{x})$의 형태에 따라 다음과 같이 분류된다.
\begin{itemize}
\item $f(\mathbf{x})$와 구속조건이 선형적이면, 선형프로그래밍 문제이다.
\item $f(\mathbf{x})$가 2차 함수이고 구속조건이 선형적이면, 2차 프로그래밍 문제이다.
\item $f(\mathbf{x})$가 선형도 2차 함수도 아니거나, 혹은 구속조건이 비선형이면 비선형 프로그래밍 문제 이다.
\end{itemize}
또한 식(\ref{eq:pt4-20})과 식(\ref{eq:pt4-21})이 포함되면 구속최적화(constrained optimization) 문제이며, 그렇지 않으면 비구속최적화(unconstrained optimization) 문제이다.

\section{비구속 최적화\\(Unconstrained Optimization)}
\subsection{황금분할법}
한 개의 비선형 방정식의 근을 구하는데 있어서의 목표는 $f(x)$가 0이 되는 $x$의 값을 찾는 것이다. 단일변수 최적화 문제(single-variable optimization)는 $f(x)$의 최대값과 최소값을 나타내는 극값인 $x$를 구하는 것이 목적이다.\\
황금분할탐색법은 단순하며 일반적으로 적용 가능한 단일변수 탐색방법이다. 이 때 근을 구하기 위해 구간법에서 학습한 이분법과 의미를 같이한다. 
\begin{equation*}
x_{r}=\frac{x_{l}+x_{u}}{2}
\end{equation*}
이분법에서와 같이 한 개의 해를 담는 구간을 먼저 정의한다. 즉, 그 구간에는 한개의 최대값이 담겨 있어야 한다. 이러한 경우를 단모드(unimodal)이라고 한다.\\
두 개의 함수값을 사용하는 것(부호의 변화를 탐색하여 근을 찾음)과 다르게 최대값의 발생여부를 탐색하기 위하여 세 개의 함수값이 필요하다. 따라서 구간 내의 또 다른 하나의 점을 선택하여야 한다. 다음으로는 네 번째 점을 구간 내에 적절하게 배치하여 최대값이 앞의 세 점 사이에 나타났는지, 나중 세 점에서 나타났는지를 확인하는 것이다. 이러한 방법이 효율적으로 진행되기 위하여 중간점들의 현명한 선택이 중요하다. 이러한 방법이 효율적으로 진행되기 위하여 중간점들의 현명한 선택이 중요하다. 이분법에서처럼 과거의 값들을 새로운 값들로 치환하여 함수값 계산을 줄이는 것이 목표이다.
\begin{align}
l_{0}&=l_{1}+l_{2}\\
\frac{l_{1}}{l_{0}}&=\frac{l_{2}}{l_{1}}
\end{align}
첫 번째 조건은 두 개의 부가적 길이인 $l_{1}$과 $l_{2}$의 합이 원래 구간의 길이가 되도록 한다. 두 번째 조건은 길이의 비가 같도록 함을 나타낸다.
\begin{equation}
\frac{l_{1}}{l_{1}+l_{2}}=\frac{l_{2}}{l_{1}}
\end{equation}
위 식의 역수를 취한 후 $R=l_{2}/l_{1}$으로 하면, 다음 식을 얻게 된다.
\begin{align}
1+R&=\frac{1}{R}\\
R^{2}+R-1&=0
\end{align}
여기서 양수의 근을 구하면 다음과 같다.
\begin{equation}
R=\frac{-1+\sqrt{1-4(-1)}}{2}=\frac{\sqrt{5}-1}{2}=0.61803\cdots
\end{equation}
이 값은 고대로부터 황금비(golden ratio)라고 알려져 왔다. 이것을 이용하여 최적값을 효율적으로 얻을 수 있기 때문에 우리가 지금까지 개념적으로 전개해온 황금분할법의 핵심요소가 된다. 이 방법을 컴퓨터에서 구현되도록 알고리즘을 유도해보자.
이 방법은 $f(x)$의 국부적 극점을 포함하는 두 개의 초기추측값 $x_{l}$과 $x_{u}$로 시작한다. 다음은 두 개의 내부점 $x_{1}$과 $x_{2}$를 황금비를 이용하여 구한다.
\begin{align*}
d&=\frac{\sqrt{5}-1}{2}\left(x_{u}-x_{l}\right)\\
x_{1}&=x_{l}+d\\
x_{2}&=x_{u}-d
\end{align*}
두 개의 내부점에서 함수값을 구하면 두 가지 결과를 얻을 수 있다.
\begin{itemize}
\item[1] $f(x_{1})>f(x_{2})$이면 $x_{2}$의 왼쪽에 있는 $x$의 영역, 즉, $x_{l}$로 부터 $x_{2}$까지의 구간을 제거할 수 있다. 왜냐하면 그 구간에는 최대값이 없기 때문이다. 이 경우 $x_{2}$가 다음 반복 시행의 새로운 $x_{l}$이 된다.
\item[2] $f(x_{2})>f(x_{1})$인 경우 $x_{1}$의 오른쪽에 있는 영역, 즉 $x_{1}$부터 $x_{u}$까지의 구간이 제거된다. 이 경우 $x_{1}$이 다음 반복시행의 새로운 $x_{u}$가 된다.
\end{itemize}
여기서 황금비 사용에 따른 실제적인 이점을 생각해 보자. 최초의 $x_{1}$과 $x_{2}$가 황금비를 따라 선택되었기 때문에 다음 반복수행시 모든 "함수값"을 다시 계산할 필요가 없다.
알고리즘을 마루리하기 위하여 새로운 $x_{1}$을 결정하여야 한다. 이것은 전과 같은 비례상수를 사용하여 얻어진다.
